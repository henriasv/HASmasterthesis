\chapter{Simulation tools}

\section{LAMMPS}

\subsection{Input files}
Lammps input files are a set of commands to be executed. Some commands are only valid in the right order. The structure of such a file is:

\begin{enumerate}
\item \tb{Initialization:} Units, processors, boundary conditions, atom\_style, pair\_style
\item \tb{Atom definition:} Either from input file (restart/data) or making a lattice or region and create a box and atoms. 
\item \tb{Settings:} 
\item \tb{Run:}	
\end{enumerate}

Most properties are already set with a standard value, which means that only non-standard properties has to be set. 

\paragraph{atom\_style}
atom\_style decides what kinds of interactions are possible to implement.

\paragraph{pair\_style}
The pair\_style interaction is somewhat misleading. What it actually means is that bonds and angles are not predefined. 

All commands ar available from the LAMMPS documentation \cite{lammps:input_commands}. A very good beginners guide for LAMMPS \cite{lammps:guide:pittsburg}