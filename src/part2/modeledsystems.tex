% file: modeledsystems.tex

\chapter{Modeled systems}
This chapter contains descriptions of the systems that I have modeled durung my Master project. 
\section{Bulk S1 hydrate}
\subsection{Elastic properties}
To my knowledge, there are no published estimates of Youngs modulus and the Poisson ratio for the TIP4P/ICE+UAM model of methane hydrates. Therefore, I seek to make crude estimates of these quantities in dynamic simulations. I apply a constant strain rate by continously rescaling particle positions in one direction during MD-simulations. The other directions are kept under a constant pressure with anisotropic barostatting. Figure \ref{fig:stress_strain_11_11_11_tip4p_ice_uam} shows the stress strain relationships and corresponding estimates of Youngs modulus for a system of 11x11x11 S1 unit cells subjected to strain rates of \SI{5d-7}{\per\femto\second} and \SI{2d-7}{\per\femto\second}. By extrapolating the results to quasistatic strain, Youngs modulus is estimated to \SI{7.1}{\giga\pascal}.

\begin{figure}
\includegraphics[width=12cm]{../figures/thesis/stress_strain_11_11_11_tip4p_ice_uam.pdf}
\caption{Stess-strain relations for a system of 11x11x11 S1 unit cells. Dashed lines indicate the region that was used to estimate Youngs modulus. Strain rates of \SI{5d-7}{\per\femto\second} (blue) and \SI{2d-7}{\per\femto\second} (red) along the x-axis.}
\label{fig:stress_strain_11_11_11_tip4p_ice_uam}
\end{figure}

\section{Vega 2010}
\section{Cracking methane hydrate with non-isotropic expansion}
In an attempt to see fracture in methane hydrates, I induce a non-isotropic pressure condition on the an S1 hydrate. In order to control the nucleation point, i make a hole in the hydrate structure. A naive approach is to remove a single S1 unit cell from the system, and see what happens when it expands. A different approch might be to cut out a sphere or some other simple geometrical shape. 


