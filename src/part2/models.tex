\chapter{A model for methane hydrates}
\label{ch:models}
With the theoretical framework of molecular dynamics in place, I go on to introduce specific potentials, and choose which one to use to model cracks in methane hydrates.  

\section{Criteria for a sufficient model}
I will model fracture of methane hydrates. If the methane hydrates are pure, i.e. no sediments nearby, it is reasonable to assume that noe chemical reactions will take place. Conversely, if the hydrate are to be modeled near for example a silicon oxide surface, chemical reactions can be important. Since chemical reactions are not expected to occur in pure hydrates, bonding potentials can be used for covalent bonds. The transferable interpotential models (TIP-models) are popular models that use bonding potentials for the OH-bonds. Since I will model fracture, the mechanical properties should be reasonably represented. I lack experience with water models, so I end up using a model that other people have shown to work. Specifically, I choose a model that was able to spontaneously nucleate methane hydrates in molecular dynamics simulations. The strategy is to use a model that successfully reproduced some feature of methane hydrates to study another feature. This means we cannot immediatly trust the model, and this study is just as much a test of whether the model I choose is suitable for studying fracture of methane hydrates.

\section{TIP4P water model}
The TIP4P/ICE was developed by \citet{Abascal2005} as a reparametrization of the TIP4P water model. TIP4P was used by \citet{Matsumoto2002} in the first successfull simulations of nucleation of ice crystals in molecular dynamics, and TIP4P/ICE was used by \citet{Walsh2009} in the first successful simulation of spontaneous methane hydrate nucleation. 

\subsection{Overview}
TIP4P is a rigid water model with 4 sites for each water molecule: One O-site, two H-sites, and a site commonly referred to as the M-site. The M-site is supposed to slightly move the oxygen charge, and it is situated on the bisector of the H-O-H angle. The parameters for the arrangement as well as the masses of hydrogen and oxygen are taken from experimental observations, and were listed in table \ref{tb:intro:h2odata}.

\begin{table}[h!tb]
\caption{Parameters for TIP4P-ICE model as in \cite{Abascal2005}.}
\begin{center}
\begin{tabular}{c|c|c}
Description & Symbol & Value \\ 
\hline
Lennard-Jones energy & $\varepsilon/k$ & \SI{100.5}{\kelvin} \\
Lennard-Jones characteristic distance & $\sigma_{OO}$ & \SI{3.155}{\angstrom} \\
Distance O-site to M-site along bisector & $d_{OM}$ & \SI{0.157}{\angstrom} \\
Hydrogen charge & $q_H$ & \SI{0.5676}{\elementarycharge} \\
\end{tabular}
\end{center}
\end{table}

\subsection{Interactions}
There are two interaction potentials for TIP4P-ICE: electrostatic interactions and Lennard-Jones (LJ) interactions, with the following functional forms:

\begin{align}
	U_{LJ} & = 4\varepsilon\left[\left(\frac{\sigma}{r_{OO}}\right)^{12} - \left(\frac{\sigma}{r_{OO}}\right)^{6}\right]
	\label{eq:part1:lennardjonespotential}
	\\
	U_{\si{\elementarycharge}} & = \frac{\si{\elementarycharge\squared}}{4\pi\varepsilon_0} \sum_{a, b} \frac{q_aq_b}{r_{ab}}
	\label{eq:part1:electrostaticpotential}
\end{align}

Since the molecules are treated as rigid, interactions are only between sites on different molecules. 

\subsection{Parameters for the TIP4P family}
TIP4P is a family of potentials, of which TIP4P/ICE is one member. The TIP4P potentials differ only in the values of their parameters. TIP4P/ICE has not been thoroughly investigated, and therefore, several simple properties such as the shear viscosity are unknown. To be able to verify my use of the TIP4P/ICE, i will do simulations with parameters from other TIP4P models for verification, and then I will work out properties for the TIP4P model. For this I need parametes for additional members of the TIP4P family. I these parameters in table \ref{tbl:tip4p_parameters}.

\begin{table}
\caption{Parameters for the different parametrizations in the TIP4P family.}
\label{tbl:tip4p_parameters}
\begin{tabular}{c|c|c|c}
Description & Symbol & TIP4P & TIP4P/2005 \\ 
\hline
Lennard-Jones energy & $\varepsilon/k$ & \SI{78.0}{\kelvin} & \SI{93.2}{\kelvin} \\
Lennard-Jones characteristic distance & $\sigma_{OO}$ & \SI{3.154}{\angstrom} & \SI{3.1589}{\angstrom} \\
Distance O-site to M-site along bisector & $d_{OM}$ & \SI{0.150}{\angstrom} & \SI{0.1546}{\angstrom} \\
Hydrogen charge & $q_H$ & \SI{0.520}{\elementarycharge} & \SI{0.5564}{\elementarycharge} \\
\end{tabular}
\end{table}

\section{Optimized Potentials for Liquid Simulations (OPLS)}

\subsection{OPLS methane}
OPLS-UA and OPLS-AA. I will choose OPLS-UA for simplicity. Another potential might be used in the future. 
United atom methane is the united atom model for methane, and is effectively a single Lennard-Jones interaction site for each methane molecule. 
\begin{equation}
	U_{LJ} = 4\varepsilon\left[\left(\frac{\sigma}{r_{MM}}\right)^{12} - \left(\frac{\sigma}{r_{MM}}\right)^{6}\right]
	\label{eq:part1:lennardjonespotentialmethane}
\end{equation}
A parameter set for this methane representation can for example be taken from \cite{Martin1998}, and are listed in \ref{tb:parameters_unitedatommethane}. 

\begin{table}
\caption{Lennard-Jones parameters for united atom methane.}
\label{tb:parameters_unitedatommethane}
\begin{center}
\begin{tabular}{c|c|c}
Description & Symbol & Value \\
\hline
Lennard-Jones energy & $\varepsilon/k_B$ & \SI{147.9}{\kelvin} \\
Lennard-Jones characteristic distance & $\sigma_{MM}$ & \SI{3.73}{\angstrom}
\end{tabular}
\end{center}
\end{table}

\section{Combining particles of different species in a model}
Having different potentials, for different aggregate particles (such as TIP4P/ICE), we want a way to combine them in a model containing both species. For electrostatic interactions, the combination rules are straightforward, as they follow from Coulombs law. For Lennard-Jones interactions, there is not such a simple combination law, and in principle, one has to fit parameters for each unique pair of particles. Luckily, it is possible to make rules that combine particles without going into an extensive parameter fitting exercise. 

\subsection{Lorentz-Berthelot combination rules}
Perhaps the simplest way to combine two species Lennard-Jones particles, is to use the following rules from two sets of parameters ($\varepsilon_{ii}$, $\sigma_{ii}$) and ($\varepsilon_{jj}$, $\sigma_{jj}$) for interactions between two particles of the same species i or j, on to a set of parameters ($\varepsilon_{ij}$, $\sigma_{ij}$) for interactions between one particle of each species:
\begin{align}
\varepsilon_{ij} = \sqrt{\varepsilon_{ii}\varepsilon_{jj}} \\
\sigma_{ij} = \frac{1}{2} \left(\sigma_{ii}+\sigma_{jj}\right)
\end{align}

\section{TIP4P-ICE + OPLS-UA methane}
We now have all the ingredients we need to model a methane hydrates: A water model, a methane model, and a way to make the water and the methane interact with each other. This model is essentially a description of the total energy of a system consisting of particles with known positions and momenta. The next step is to use molecular dynamics to explore the properties of this model, and to see if we can use this model to learn something about methane hydrate.

\section{A system for studying fracture of methane hydrates}
It was suggested by \citet{Ning2012} to find basic mechanical properties like Young's modulus and Poisson's ratio and the fracture toughness of methane hydrates using molecular dynamics. Particulary, they propose to study methane hydrates with a dislocation subjected to tensile stress. Since I have not found any studies on the tensile strength of methane hydrates using molecular dynamics, I choose to model a system close to the simplest case in fracture mechanics: A rectancular prismatic piece of methane hydrate with an artificial elliptical prismatic crack spanning its z-direction. This system will be subjected to tensile stress normal to the crack plane. Apart from the crack that is carved out, the hydrate lattice will be fully occupied with methane molecules. Studies of partly occupied lattices are left for future work.

