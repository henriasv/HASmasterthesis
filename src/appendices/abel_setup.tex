\chapter{Details of using the Abel computing cluster}

\section{Compiling lammps on the Abel computing cluster}
It is usually quite straightforward to install the main features of LAMMPS. However, if one is to use special features such as GPU packages or th Intel Xeon Phi, it is more complicated... 	

To compile LAMMPS on the Abel computer cluster, one has to load the intel compiler and mpi modules, and then follow the build instructions from the LAMMPS documentation. The program used for simulations going into this thesis, is compiled with the following command:
\begin{lstlisting}[language=Bash]
# Script to build lammps with openMP and Intel on Abel (November 21. 2014)
module load intel
module load intelmpi.intel
make yes-user-intel
make yes-user-omp
make yes-kspace
make yes-replica
make yes-molecule
make yes-rigid
make intel_cpu
\end{lstlisting}
These commands are run from {\tt src} in the LAMMPS install folder, which was extracted from a tarball.

\section{Rsync}

\section{Submitting jobs}
When expanding templates, a corresponding job script is also generated. It typically looks like:
\begin{lstlisting}[language=Bash]
#!/bin/bash
# Job name:
#SBATCH --job-name=s1_hydrate_crack.in2015-02-12T11:17:08.271798
# Project: 
#SBATCH --account=nn9272k
# Wall clock limit:
#SBATCH --time=\'10:00:00\'
#SBATCH --mem-per-cpu=4000M
# CPUs:
#SBATCH --nodes=18 --ntasks-per-node=1 --cpus-per-task=15
module purge
module load intel
module load intelmpi.intel
mpirun -np 18 lmp_intel_cpu -package omp 15 -suffix omp -in s1_hydrate_crack.in
\end{lstlisting}