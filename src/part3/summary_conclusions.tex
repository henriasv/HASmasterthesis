\chapter{Summary and conclusions}
\label{ch:summary_conclusions}
In chapter \ref{ch:state_of_the_science}, I presented the current state of the science of methane hydrates, and noticed that there have been no molecular dynamics studies on fracture of methane hydrates. I also outlined a selection of questions that I wanted to answer. The goal of this thesis has been to develop a protocol for imposing and studying fracture in methane hydrates, and to use this protocol to characterize the fracture and improve the understanding of this process. 

In this chapter, I summarize my findings and the conclusions I draw from them.

%%%%%%%%%%%%%%%%%%%%%%%%%%%%%%%%%%%%%%%%%%%%%%%%%%%
\section{Summary and conclusions}
%%%%%%%%%%%%%%%%%%%%%%%%%%%%%%%%%%%%%%%%%%%%%%%%%%%
% Summary of the work and the working process, experiences
% Description of how the system behave
% Conclusions

I have modeled systems of pure methane hydrates of the sI structure using LAMMPS. I have introduced artificial flaws -- elliptical prismatic cracks -- in the systems, and subjected them to tensile strain. 

\textit{Fracture toughness and brittleness:} When subjected to a stress intensity factor of $\sim 0.06$ MPam$^{\frac{1}{2}}$, the hydrate failed, and a crack started propagating. For the particular geometry of my systems, this happed at strains of $\sim 0.05$. The system was unable to deform plastically prior to failure, and when crack propagation started, the crack went all the way to the periodic boundaries of the system -- the methane hydrate appeared brittle. The brittleness was further confirmed by the strain energy required to propagate a crack being almost equal to the energy associated with the crack surface. 

\textit{Fracture of large systems:} Even though the required strain for failure seemed quite well defined, the rupture was not immediate at this strain level. There was a waiting time from straining until rupture. It seems, but based on a limited amount of statistics, that this waiting time was more predictable for higher than for lower temperatures -- at least in the sense that a lower strain gave a longer waiting time. This observation likely resulted from the slow dissociation process (melting) being more prominent at higher temperatures.  Furthermore, the applied stress--critical crack length relation seems to more robustly characterize failure than a waiting time after applying a certain strain: The stress--crack length curve after loading and before rapid failure had a well-defined slope, and the point where the transition from slow to fast crack propagation happened placed itself close to the theoretical prediction by LEFM. I believe that the fracture toughness and the ductility of methane hydrates under tensile loading is highly temperature-dependent because of this effect. In further investigations, it will probably be fruitful to characterize quantitatively the relation between the stress on the hydrate close to the crack and the dissociation rates. 

\textit{Methane molecules in the fracture:} When the hydrate failed, methane was almost immediately released, but only from a narrow region of the wall close to the crack pore space. Methane from about half the width of a unit cell in the system was released.

\textit{Crack instabilities:} Crack instabilities have not been prominent, even though the crack speeds were measured to be within the range where crack instabilities are expected to exist. This could very well be because the modeled system was too short for instabilities to develop.

\textit{Fracture energy:} When calculating the fracture energy, I found that the full potential energy of the system was a bad estimate of the free energy, because most of the energy introduced during straining was entropic energy supplied by the thermostat. The strain energy had to be calculated by explicitly integrating the stress with the strain. 

\textit{Large-scale simulations:} On the more practical part, I have obtained experience with performing molecular dynamics simulations. Quite early, I decided that I wanted to make a system for setting up and running simulations that was more sophisticated than editing individual input files. I ended up creating a template format and a Python code to expand the templates. The templates closely resemble runnable LAMMPS input files, but with some decoration to be able to customize the simulations. This made each simulation more reliable, since the template was already tested and refined. The templates are explained in appendix \ref{ch:templates}.


%%%%%%%%%%%%%%%%%%%%%%%%%%%%%%%%%%%%%
\section{Discussion}
%%%%%%%%%%%%%%%%%%%%%%%%%%%%%%%%%%%%%
% Limitations of the approach -- Method and systems
% Usefulness of conclusions

The most profound limitation in molecular dynamics is the representation of the inter-atomic interactions by classical potentials. Even with infinite computer power, the value of the simulations rely heavily on the quality and applicability of the potentials, and the modeling of water turns out to be extra challenging. It is impossible to get the exactly right behavior without including quantum effects. But the point of doing molecular dynamics instead of quantum methods it to look at processes whose time- and spatial scales are not available with quantum mechanics methods -- we must deal with the limitations of molecular dynamics. Properties that are not easy to fit empirically, such as the behavior under high stresses and strains, can be hard to assess. Because of this profound limitation, the robustness of results of molecular dynamics simulations are clearly limited. For my specific simulations, the quantitative results are probably way off, since the potentials were not specifically developed to reproduce the correct mechanical properties or the growth- and dissociation rates of methane hydrates. Indeed, while Young's modulus showed good correspondence with experiments, the Poisson's rate was far from the experimental value (0.41 in my work vs. 0.317 from experiments). However, since the potential can comfort the structure of the methane hydrate, and is able to spontaneously grow the hydrate, the qualitative properties arising from the model are probably of interest. Particularly, governing mechanisms in the failure of methane hydrates were identified: The slow dissociation (melting) is important for the fast crack propagation. These kinds of mechanisms,  involving the interplay between the external stress on a crack and the behavior of individual molecules close to the crack and the crack tip, can probably be usefully identified from molecular dynamics simulations.

Further on the limitations, the systems I modeled were very clean, in at least four ways: 
\begin{itemize}
\item The hydrate structure was a pure sI crystal. 
\item There were no impurities -- particles of other types than water and methane. Impurities will probably exist in an experimental or geological setting, both in form of other quest molecules and water soluble substances like salt. 
\item The loading condition was purely tensile, and normal to the column of unit cells, favoring a single crack direction.
\item The initial crack was an elliptical prism spanning the whole $z$-direction. This was done for efficient comparison with linear elastic fracture mechanics, but it can also result in the system behaving almost two-dimensional. 
\end{itemize}
It is possible that the brittleness I observe in my simulations can be partly attributed to these purities, and not just the short time-scale.

Molecular dynamics simulations are limited in time and space, because of limited amounts of computing power. That means some processes cannot be studied. There might, an probably do, exist processes that are too slow to be captured efficiently in MD simulations. The straining rates of my simulations were set high because they had to, not because the high straining rates were experimentally or geologically realistic. Thus the state of the system subjected to more realistic loading rates may be different than the state of the systems I modeled. That might change the behavior during crack initiation. The brittle appearance of the methane hydrate in my simulations stands in contrast to the experimental observations mentioned in chapter \ref{ch:state_of_the_science}, where methane hydrates showed a great ability to deform plastically. This may very well be a result of the high strain rate, disallowing the system to reorganize to adapt to the strain.


%%%%%%%%%%%%%%%%%%%%%%%%%%%%%%%%%%%%
\section{Outlook}
%%%%%%%%%%%%%%%%%%%%%%%%%%%%%%%%%%%%
% What is interesting to other people?
% How to handle the limitations
% Practical considerations, such as implementations that can be done

From a practical point of view, the most interesting part is probably how dissociation and stability of methane hydrates can be predicted and controlled. Predictions are necessary to assess the safety of extracting methane from hydrates, and to find out whether methane hydrates can pose risks to life on earth. Controlling the dissociation is crucial to be able to produce methane from the hydrates. Reaching a level of sophistication in the description of methane hydrates where this is possible is probably closely tied to answering more fundamental scientific questions. The questions of chapter \ref{sec:research_questions} are such questions. Based on the knowledge I have obtained during the work with this thesis, I propose topics for future work, that can both build on my work, and contribute to answering the more general questions regarding how methane hydrates can impact peoples lives:


\begin{itemize}
\item Further investigations to confirm the two stages of dissociation: Thermally activated crack growth (slow) and strain-activate crack growth (fast). Particularly, a study of the dissociation properties under the extreme conditions near the crack tip would be interesting to elucidate the interplay between the fast and slow fracture mechanisms.
\item Other potentials. A water potential should be parametrized with a focus on reproducing the correct mechanical behavior, and at the same time have realistic growth- and dissociation rates of methane hydrates. Profound problems concerning the modeling should be taken into account, and first-principles studies of water under high stress and strain configurations can probably contribute to potentials that can be trusted for fracture simulations. 
\item Longer simulations of smaller systems can serve as a tool to identify processes on longer timescales. Simulations of hydrate growth had to last for microseconds, which means this timescale is important. My simulations only lasted for around a nanosecond.
\end{itemize}

I have modeled a very simple system. Further investigations should probably attack a broader set of systems:

\begin{itemize}
\item Less ordered structures, i.e. not replications of a unit cell.
\item Changing the hydration number. Molecular dynamics simulations can shed light on how the cage occupancy influences the mechanical properties of methane hydrates.
\item Defects of different geometries.
\item Other loading modes. This point is in relation to one of the points below, since shear loading requires implementation of a new feature in LAMMPS.
\end{itemize}
%

On the more practical part, I propose some features that would be convenient to have in the simulation package:
\begin{itemize}
\item TIP4P water with P$^3$M and triclinic box. This combination is currently not supported in LAMMPS, and that disallows imposing shear stress by shearing the simulation box (the method of \citet{Parrinello1981}).
\item LAMMPS extensions to perform more of the analysis during simulation, to be able to analyze cracks with high time resolution.
\end{itemize}
 
\section{Ending remarks}
Knowledge on a molecular level only become practically interesting if is is possible to scale it to macroscopic quantities it in some way or another. Purely scientifically, the role of molecular dynamics calculations on methane hydrates is probably to provide clues about what processes should be included in upscaled models -- models that gain access to longer and larger scales by ignoring more details than the lower-scale model. This is crucial when considering the important question whose answers can impact peoples lives: Can, and how can, methane hydrates be used as an energy resource? Can methane hydrates cause uncontrollable climate change, or devastating underwater landslides, and can we prevent it? 