\chapter{Summary and conclusions}
\label{ch:summary_conclusions}
\section{Summary of findings}
Notes for the summary
\begin{itemize}
\item Methane hydrates are brittle in my simulations, but they are quite ductile in experiments.
\item Methane is quickly released during fracture.
\item Someting on dissociation-activated fracture.
\item Reading data from disk is a bottleneck when analyzing data post-simulation. 
\end{itemize}

\section{Conclusion}
Notes for the conclusion
\begin{itemize}
\item Since $G_c = 2\gamma_s$ the methane hydrates are very brittle in my simulations. (the details of the methane-water interaction might be crucial). OPLS-AA should be tested.
\item I speculate that hydrate dissociation in the presence of fracture is highly dependent on slow dissociation, which means the fracture toughness likely highly depend on the pressure and temperature, which drive the non-fracture-induced dissociation. This fits well with experimental observations, namely that the compressional strength increases with decreasing temperature, increasing confining pressure and increasing density.
\item It is actually quite remarkable that temperature is so important for fracture at these low temperatures.
\end{itemize}

Regarding the robustness of my results, I think that a twofold approach is necessary. First, LAMMPS is well tested, and since I ended up using mostly the core features of LAMMPS. This package should be stable, and since I only use TIP4P water and a pure Lennard-Jones methane, it is highly likely that the simulation actually simulate the model I want it to simulate. Then is the question of the model itself. And the model is probably not well suited to make accurate predictions of the fracture properties of methane hydrates. But since TIP4P/Ice + UAM has proved to reasonably calculate growth and dissociation of methane hydrates, I think it is reasonable to conclude that the fracture initiation is highly influenced by slow dissociation processes. Confirmation using other potentials and looking more closely on effects of the methane in the initial crack will probably be needed to conclude definitely.

\section{Outlook}
\begin{itemize}
\item Furter investigations to confirm the two stages of dissociation: Thermally activated crack growth (slow) and strain-activate crack growth (fast). Particularly, a study of the dissociation properties under the extreme conditions near the crack tip would be interesting to eludicate the interplay between the fast and slow fracture mechanisms.
\item Other potentials -- the problem with high poisson ratio. It might be a good idea to parametrize a water model that reproduces reasonable elastic features of the empty hydrate lattice, and then, using ab-initio calculations fit a methane-water potential.
\item Less ordered structures.
\item Changing the hydration number. Molecular dynamics simulations can shed light on how the cage occupancy influence the mechanical properties of methane hydtates.
\item Defects of different geometries.
\item Other loading modes.
\item Experimental ductility of methane hydrates: Is there a timescale at which methane hydrates become ductile, or perhaps heterogeneities can make sure that methane hydrates can sustain a global strain comparable to the yield strain of the perfect crystal, but with pores of methane forming inside the material. 
\item Develop a fracture analysis package for LAMMPS: Since so large amounts of data are needed for fracture analysis, it is probably a good idea to do big analyses live.
\item Implement support for TIP4P with P$^3$M and triclinic box to control shear stress.
\item Reparametrize the TIP4P model for the study of methane hydrates.
\end{itemize}