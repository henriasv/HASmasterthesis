\chapter{The science of Methane Hydrates}
Methane hydrates are clathrate compounds, which means they consist of host molecules forming a lattice that traps guest molecules. In the particular case of methane hydrates, water molecules form cages that host methane molecules. Ther cage structure is mainly comprised by pentagonal and hexagonal faces, and they can form regular cage stuctures.



\section{The Water and Methane Molecules}
\begin{table}[h!tb]
\caption{Experimental data for the water molecule.}
\label{tb:intro:h2odata}
\begin{center}
\begin{tabular}{c|c|c}
Description & Symbol & Value \\
\hline
H-O-H angle & $\theta$ & \SI{104.52}{\degree} \\
Distance O-H & $d_{OH}$ & \SI{0.9572}{\angstrom} \\
\end{tabular}
\end{center}
\end{table}


\section{Elastic properties, strength and toughness of methane hydrates}

\section{Molecular Modeling of Methane Hydrates}

\section{Molecular Modeling of Fracture}

\section{Research questions}
\begin{itemize}
\item What is the fracture toughness of methane hydrate?
\item What characterizes sub-critical fractures in methane hydrates? 
\end{itemize}