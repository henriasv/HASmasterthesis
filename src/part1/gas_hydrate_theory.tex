\chapter{State of the science of methane hydrates}
\label{ch:state_of_the_science}
Methane hydrates are clathrate compounds, which means they consist of host molecules forming a lattice that traps guest molecules. Special to clathrate hydrates, which are clathrates there water form the lattice, is that the structure is stabilized by the guests, and would collapse into regular ice or liquid water without guests present. Metane hydrates are among the more prevalent clathrate hydrates, and are formed by water molecules providing cages that host methane molecules. The most common cage structures are comprised of pentagonal and hexagonal faces forming regular cage stuctures. These structures can be described as replications of relatively simple unit cells. This work focus on methane hydrates, but since other clathrate hydrates have also been researched - and are relevant - they may be briefly discussed.


\section{Molecular structure}
Different hydrate structures form based on the size of the guest molecules. These structures are characterized by what kinds of cages they contain. The cages are described with the notation $x^y$ where $x$ is the number of faces of a cetain type, and $y$ describes the type of face by the number of corners on that face of that type. The most common structures for clathrates with one type of guest molecule are the so called structure I (sI) and structure II (sII). These structures, along with the cages that form them, are illustrated in figure \ref{fig:methane_hydrate_structure}. The sI structure contains $5^{12}$ and $5^{12}6^2$ cages in ratio $1:3$, the sII structure contains $5^{12}$ and $5^{12}6^4$ cages in ratio $2:1$.  

Small guest molecules, such as methane and ethane, usually form sI hydrates if the conditions for hydrate formation are met \cite{Hester2009}. However, that is only certain if the hydrate is either purely methane or purely ethane. For mictures of methane and ethane forming gas hydrates, either sI or sII can be formed, depending on the relative amounts of methane and ethane \cite{Subramanian20001981}. 

Not all cages need to be occupied by a guest molecule, so a pure methane hydrate sample can, in addition to its cage structure, be characterized by its cage occupancy. It is common to use the hydrate number to describe cage occupancy, which for methane hydrate looks like:

\begin{equation}
	\mathrm{CH_4} \cdot n_w \mathrm{H_2O}
\end{equation}

Where $n_w$ is the hydrate number. For a fully occupied sI hydrate, the hydrate number would be $n_w = 5.75$. Hydrate numbers have been reported both for laboratory grown methane hydrates and for natural occuring ones. Hydrates grown in laboratory have shown high cage occypancy both with excess water and excess methane, altohugh excess methane yield the highest occupancy. \citet{Circone2005} report values within 5.9 to 6.1 for a relatively wide range of growth conditions: Pressures from 1.8 to \SI{9.6}{\mega\pascal} and temperatures from 263 to \SI{287}{\kelvin}. 

\begin{figure}
\includegraphics[width=15cm]{../pictures/hydrate_structures.png}
\caption{Cage structures of single guest methane hydrates occuring in nature. Both structure 1 (sI) and structure 2 (sII) contain the $5^{12}$ cage. Reprinted from \citet{Barnes2013} with permission from Elsevier.}
\label{fig:methane_hydrate_structure}
\end{figure}


\section{Mechanical properties from experiments}
Since methane hydrates come in several forms, it is hard to say someting general about the mechanical properties. In a review paper from 2012, \citet{Ning2012} state:
\begin{quotation}
Few mechanical properties are reported , and their measurements are difficult, partly because it is almost impossible to obtain pure hydrate samples.
\end{quotation}

This must be taken into account when going through the experimental results on mechanical properties of methane hydrates. It also means that experimental results probly will not be directly comparable to molecular models of pure samples.

Much of the research has been done on hydrate-bearing sediments, and functional relationships have been proposed that relate the strength of a hydrate-bearing sediment to the pure hydrate strength and the hydrate saturation (the amount of hydrate in the sediment). But since the strength of the mechanical properties of the hydrate are uncertain, the assessment of such relationships is hard. \citet{Ning2012} consider the mechanical properties of pure hydrates essential for understanding the mechanics of hydrate-bearing sediments. Furthermore, almost all strength tests on methane hydrates, tests where the hydrate is subjected to some stress to break it, are axial compression tests, which leaves a limited basis for comparison with simulations. 

\subsection{Typical experimental setup}
A main challenge when doing experiments on methane hydrates is that samples of methane hydrate are not stable in room temperature and atmospheric pressure. Therefore, experimental equipment for the study of mechanical properties of hydrates consist of two main parts: A hydrate formation unit, and a measurement unit, so that the hydrate doesn't need to be removed from its stability conditions during measurement. Sometimes, the measurement unit is actually an axial compression chamber, so that axial tests can be done.

\subsection{Experimental results}
Even though we cannot guarantee the purity of methane hydrate samples used in experiments, there are still robust findings that are valuable pointers for numerical investigations. \citet{Waite2000} measured mechanical properties using the compressional and shear wave speeds, and assuming the sample to be isotropic and homogeneous. Their results are given in table \ref{tbl:si_mech_exp}.

\begin{table}
\caption{Some mechanical properties of sI methane hydrate as reported in \cite{Waite2000}.}
\label{tbl:si_mech_exp}
\centering
\begin{tabular}{c|c}
Property & Value \\
\hline
$V_p$ & \SI{3650\pm 50}{\meter\per\second} \\
$V_s$ & \SI{1890\pm 30}{\meter\per\second} \\
Poisson's ratio & \SI{0.317 \pm 0.006}{} \\
Shear Modulus & \SI{3.2\pm0.1}{\giga\pascal} \\
Isothermal Young's modulus & \SI{7.8\pm 0.3}{\giga\pascal}
\end{tabular}
\end{table}

In the review by \citet{Ning2012}, it is claimed, based on numerous experiments they reviewed, that the tendency for the compressive strength of methane hydrates, is that it increases with increasing confining pressure and decreasing temperature. But it is also stated that all the other mechanical properties depend highly on the temperature, the pressure, the cage occupancy etc, which means results from experiments under slightly different conditions are hard to compare. A surprising observation with regards to compressional strength, is that methane hydrates exhibit strain-hardening for compressional strains as high as around 15-20 \% \cite{Durham2003, Stern1998}, which is very high compared to regular water ice.

\section{Models of water}
Water is difficult to model, and many models are developed in the pursuit of understanding water in all its phases. 

One can represent the water molecule in many different ways, but the most common picture of the water molecule is the picuture of two hydrogen atoms connected to an oxygen atom. In this representation, the parameters for the distance between the oxyygen and the hydrogen and the angle between them have been foun experimentally. The values are given in table \ref{tb:intro:h2odata}. Another representation of the water molecule is by its wavefunction, which can be visualized for example by the electron density around the nuclei of the molecule. Such a representation can be derived from quantum mechanics. Both representations are illustrated in figure \ref{fig:water_molecule}. The quantum mechanic description is too complex to be directly applied in molecular dynamics, so the molecular dynamics representation must be some variation over the connected-particlies picture. 


@1933-model @TIP-models.  \citet{Vega2011} review rigid non-polarizable models.

\begin{figure}
\begin{minipage}{\textwidth}
\subcaptionbox{The water molecule represented with ints individual atoms connected with bonds.}{
\includegraphics[width=0.52\linewidth]{../pictures/h2o_molecule.pdf}}
\subcaptionbox{Electron density representation of the water molecule. The density is caculated with the Hartree-Fock method. Taken from my simulations for a course at the University of Oslo.}{
\includegraphics[width=0.48\linewidth]{../pictures/h2o_hf_density.png}}
\end{minipage}
\caption{Two pictures of the water molecule.}
\label{fig:water_molecule}
\end{figure}

\begin{table}[h!tb]
\caption{Experimental data for the water molecule.}
\label{tb:intro:h2odata}
\begin{center}
\begin{tabular}{c|c|c}
Description & Symbol & Value \\
\hline
H-O-H angle & $\theta$ & \SI{104.52}{\degree} \\
Distance O-H & $d_{OH}$ & \SI{0.9572}{\angstrom} \\
\end{tabular}
\end{center}
\end{table}

\section{Molecular dynamics modeling of methane hydrates}
Due to the lack of good experimental results, numerical modeling can be important to investigate methane hydrates. One of the ways to model them, is through molecular dynamics simulation, where trajectories of individual atoms are calculated using classical equations of motion. 

There are essentially two choices to be made when designing a molecular dynamics simulation: What potential models to use, and what system to simulate. By system, I mean both the initial condition and the conditions during simulation (temperature, pressure, etc.).

For the potential models of methane hydrates, there are two common strategies: All atom potentials and united atom potentials. In the all atom potentials, methane and water are represented by all of its atoms. Interactions between atoms belonging to the same molecule are bonding, and interactions between atoms belonging to different atoms are non-bonding. In the united atom potentials, all atoms in a molecule are represented by one particle. Interactions between molecules in united atom representations are non-bonding, and the functional form of the potential is usually more complicated than for the all atom models. 

In the first successful simulation of methane hydrate nucleation, \citet{Walsh2009} used a combination of these strategies. The water model was an all-atom model, TIP4P/Ice (described in detail in chapter \ref{ch:models}), and the methane model was a united-atom model, United-atom methane, which is a pure Lennard-Jones potential. Figure \ref{fig:methane_hydrate_growth} shows illustrations of the formation process.  Later, \citet{Jacobson2010b} proposed a coarse-grained model using the united-atom approach both on the water model and the methane model. This model has been used in several studies looking at nucleation and growth of methane hydrates. 

Potentials will be furter discussed in chapter \ref{ch:models}.

\begin{figure}
\centering
\includegraphics[width=\textwidth]{../pictures/methane_hydrate_growth.pdf}
\caption{Nucleation and growth of methane hydrates. From \citet{Walsh2009}. Reprinted with permission from AAAS.}
\label{fig:methane_hydrate_growth}
\end{figure}

\begin{comment}
\citet{Ning2010} calculate the bulk modulus of sI methane hydrates using several water models and methane models.

\citet{English2015} review molecular modeling of methane hydrates. I will summarize two things they mention: Dissociation and molecular-simulation approaches. They don't cite anyone studying it in conjunction with fracture. 

Chill+ identification of clathrate cages \citet{Nguyen}.
\end{comment}



\section{Quantum mechanical calculations on methane hydrates}
The properties of the sI hydrate cage has been calculated using DFT-analysis. The results differ quite a lot from the experimental values, but since the experimental values are uncertain, it is hard to assess the value of the DFT-analysis.

There have also been some efforts on fitting a methane-water potentials using ab initio quantum mechanical methods. A popular metod is by second order Møller--Plesset perturbation theory (MP2). \citet{Anderson2004} used this to develop a potential for argon--water and methane--water interactons and applied the methane--water results on methane hydrates. They fix the internal configuration of each molecule, i.e. the hydrogen positions relative to the oxygen in water and the hydrogen positions relative to carbon in methane. Then they map the six-dimensional energy surface of the methane--water interaction. The dimensions are the C--O distance and five angles describing the relative rotations of the molecules. One of the conclusions is that the Exponential--6-potential better represents the H$_4$\tb{C}--\tb{O}H$_2$ interaction than the Lennard--Jones potential. That study does not cover the interactions between particles of the same species (eg. water--water).


\section{Molecular dynamics modeling of fracture}
A lot has been done on characterizing fracture of relatively simple crystal lattices such as the FCC-lattice. \citet[ch. 6]{doi:10.1142/9789812773326_0001} reviews dynamical fracture of homegeneous lattices. An interesting result is that cracks become unstable at a certain crack velocity. This velocity depends not only on the Rayleigh wave speed, but also on hyperelastic (non-linear elastic) properties of the material. Stable, low speed cracks produce mirror interfaces, whereas unstable cracks form either mist (flat but rough) or hackle regions (rough) on the crack surface. In the studies cited in \citet[ch. 6]{doi:10.1142/9789812773326_0001}, tesile strain is applied on the boundary by forcing the position of the atoms in the edge planes parallell to the crack plane. 

\citet{Hantal2014} studied the fracture toughness of illite using clayFF and reaxFF. The actual fracture propagation was not studied -- only the fracture toughness and the stress--strain curves. In that study, the tensile strain was applied in discrete steps by expanding the simulation box normal to the crack plane and remapping all particle positions.

I have not been able to find any studies on molecular dynamics simulations of fracture in methane hydrates.


\section{Research questions}
The reality that I have not been able to find any studies on fracture of methane hydrates using molecular dynamics limits the expectations of what can be done during a master project. Below, I name a few questions concerning methane hydrates that seem within reach to address:

\begin{enumerate}
\item What is the fracture toughness of methane hydrates in simulations? This question is not only interesting to compare with experiments, but to complement experiments. Fracture toughness of matarials are usually estimated with standardized mechanical tests, but since such tests are hard to perform on methane hydrates, it is possible that simulation results can complement experimental results, and be important for instance as the pure-hydrate strength going into calculations of the strength of hydrate-bearing sediments. 
\item Is pure methane hydrate ductile or brittle, and is the brittle- or ductileness dependant on the strain rate the sample is subjected to?
\item What does the fracture surface look like? Is it mirror-like or hackle-like? Does the fracture surface develop in time after fracture propagation?
\item How much methane hydrate is dissociated during fracture? How much methane is freed?
\item How predictable are the fracture properties of methane hydrates predictable? Is it such that for a given stress or strain applied to a piece of methane hydrate, the critial stress/strain or the time it takes before it starts cracking can be accurately estimated? Or is it a statisical processes with a wide waiting-time distribution that govern the fracture initiation -- or even propagation?
\item Compressional strain hardening up to almost 20 \% strain. How can this be explained?
\end{enumerate}

I will produce novel results and insights regarding molecular dynamics modeling of methane hydrate fractures based on these questions. Question 1 and 2 will be answered quite conclusively (but only within the model I choose). Question 3 will be discussed along with question 4, offering some descriptive results but little insight. Question 5 results in a claim that I hope to be able to verify in the future. Question 5 will be ignored.

There are also questions regarding how to model methane hydrates, and event within molecular dynamics modeling of methane hydrates, little is actually known -- especially when it comes to fracture. 

\begin{enumerate}
\item What potential best reproduce fracture of methane hydrates?
\item How may cracks be triggered in simulation if the results are going to say someting about reality?
\end{enumerate}

My thesis is not about potentials, but I think it appropriate to list it, since understanding the limitations of the potentials is essential when doing molecular dynamics, and since I have to choose a potential. Question 2 will be touched, because imposing fracture forces me to choose or find a way to do it. 
