\chapter{State of the science of methane hydrates}
\label{ch:state_of_the_science}
Methane hydrates are clathrate compounds, which means they consist of host molecules forming a lattice that traps guest molecules. Special to clathrate hydrates, which are clathrates there water form the lattice, is that the structure is stabilized by the guests, and would collapse into regular ice or liquid water without guests present. Metane hydrates are among the more prevalent clathrate hydrates, and are formed by water molecules providing cages that host methane molecules. The most common cage structures are comprised of pentagonal and hexagonal faces forming regular cage stuctures. These structures can be described as replications of relatively simple unit cells. This work focus on methane hydrates, but since other clathrate hydrates have also been researched - and are relevant - they may be briefly discussed.


\section{Molecular structure}
Different hydrate structures form based on the size of the guest molecules. These structures are characterized by what kinds of cages they contain. The cages are described with the notation $x^y$ where $x$ is the number of faces of a cetain type, and $y$ describes the type of face by the number of corners on that face of that type. The most common structures for clathrates with one type of guest molecule are the so called structure I (sI) and structure II (sII). These structures, along with the cages that form them, are illustrated in figure \ref{fig:methane_hydrate_structure}. The sI structure contains $5^{12}$ and $5^{12}6^2$ cages in ratio $1:3$, the sII structure contains $5^{12}$ and $5^{12}6^4$ cages in ratio $2:1$.  

Small guest molecules, such as methane and ethane, usually form sI hydrates if the conditions for hydrate formation are met \cite{Hester2009}. However, that is only certain if the hydrate is either purely methane or purely ethane. For mictures of methane and ethane forming gas hydrates, either sI or sII can be formed, depending on the relative amounts of methane and ethane \cite{Subramanian20001981}. 

Not all cages need to be occupied by a guest molecule, so a pure methane hydrate sample can, in addition to its cage structure, be characterized by its cage occupancy. It is common to use the hydrate number to describe cage occupancy, which for methane hydrate looks like:

\begin{equation}
	\mathrm{CH_4} \cdot n_w \mathrm{H_2O}
\end{equation}

Where $n_w$ is the hydrate number. For a fully occupied sI hydrate, the hydrate number would be $n_w = 5.75$. Hydrate numbers have been reported both for laboratory grown methane hydrates and for natural occuring ones. Hydrates grown in laboratory have shown high cage occypancy both with excess water and excess methane, altohugh excess methane yield the highest occupancy. \citet{Circone2005} report values within 5.9 to 6.1 for a relatively wide range of growth conditions: Pressures from 1.8 to \SI{9.6}{\mega\pascal} and temperatures from 263 to \SI{287}{\kelvin}. 

\begin{figure}
\includegraphics[width=15cm]{../pictures/hydrate_structures.png}
\caption{Cage structures of single guest methane hydrates occuring in nature. Both structure 1 (sI) and structure 2 (sII) contain the $5^{12}$ cage. Reprinted from \citet{Barnes2013} with permission from Elsevier.}
\label{fig:methane_hydrate_structure}
\end{figure}


\section{Mechanical properties from experiments}
Since methane hydrates come in several forms, it is hard to say someting general about the mechanical properties. In a review paper from 2012, \citet{Ning2012} state:
\begin{quotation}
Few mechanical properties are reported , and their measurements are difficult, partly because it is almost impossible to obtain pure hydrate samples.
\end{quotation}

This must be taken into account when going through the experimental results on mechanical properties of methane hydrates. It also means that experimental results probly will not be directly comparable to molecular models of pure samples.

Much of the research has been done on hydrate-bearing sediments, and functional relationships have been proposed that relate the strength of a hydrate-bearing sediment to the pure hydrate strength and the hydrate saturation (the amount of hydrate in the sediment). But since the strength of the mechanical properties of the hydrate are uncertain, the assessment of such relationships is hard. \citet{Ning2012} consider the mechanical properties of pure hydrates essential for understanding the mechanics of hydrate-bearing sediments. Furthermore, almost all strength tests on methane hydrates, tests where the hydrate is subjected to some stress to break it, are axial compression tests, which leaves a limited basis for comparison with simulations. 

\subsection{Typical experimental setup}
A main challenge when doing experiments on methane hydrates is that samples of methane hydrate are not stable in room temperature and atmospheric pressure. Therefore, experimental equipment for the study of mechanical properties of hydrates consist of two main parts: A hydrate formation unit, and a measurement unit, so that the hydrate doesn't need to be removed from its stability conditions during measurement. Sometimes, the measurement unit is actually an axial compression chamber, so that axial tests can be done.

\subsection{Experimental results}
Even though we cannot guarantee the purity of methane hydrate samples used in experiments, there are still robust findings that are valuable pointers for numerical investigations. \citet{Waite2000} measured mechanical properties using the compressional and shear wave speeds, and assuming the sample to be isotropic and homogeneous. Their results are given in table \ref{tbl:si_mech_exp}.

\begin{table}
\caption{Some mechanical properties of sI methane hydrate as reported in \cite{Waite2000}.}
\label{tbl:si_mech_exp}
\centering
\begin{tabular}{c|c}
Property & Value \\
\hline
$V_p$ & \SI{3650\pm 50}{\meter\per\second} \\
$V_s$ & \SI{1890\pm 30}{\meter\per\second} \\
Poisson's ratio & \SI{0.317 \pm 0.006}{} \\
Shear Modulus & \SI{3.2\pm0.1}{\giga\pascal} \\
Isothermal Young's modulus & \SI{7.8\pm 0.3}{\giga\pascal}
\end{tabular}
\end{table}

The tendency for the compressive strength of methane hydrates, is that it increases with increasing confining pressure and decreasing temperature. This claim is made by \citet{Ning2012}, based on numerous experiments they reviewed. However, this picture is complicated, and \citet{Ning2012} also say that all the other mechanical properties depend highly on the temperature, the pressure, the cage occupancy etc.
A surprising observation with regards to compressional strength, is that methane hydrates exhibit strain-hardening for compressional strains as high as around 15-20 \% \cite{Durham2003, Stern1998}, which is very high compared to regular water ice.

\section{Models of water}
Water is difficult to model, and many models are developed in the pursuit of understanding water in all its phases. 

One can represent the water molecule in many different ways, but the most common picture of the water molecule is the picuture of two hydrogen atoms connected to an oxygen atom. In this representation, the parameters for the distance between the oxyygen and the hydrogen and the angle between them have been foun experimentally. The values are given in table \ref{tb:intro:h2odata}. Another representation of the water molecule is by its wavefunction, which can be visualized for example by the electron density around the nuclei of the molecule. Such a representation can be derived from quantum mechanics. Both representations are illustrated in figure \ref{fig:water_molecule}. The quantum mechanic description is too complex to be directly applied in molecular dynamics, so the molecular dynamics representation must be some variation over the connected-particlies picture. 


@1933-model @TIP-models.  \citet{Vega2011} review rigid non-polarizable models.

\begin{figure}
\begin{minipage}{\textwidth}
\subcaptionbox{The water molecule represented with ints individual atoms connected with bonds.}{
\includegraphics[width=0.52\linewidth]{../pictures/h2o_molecule.pdf}}
\subcaptionbox{Electron density representation of the water molecule. The density is caculated with the Hartree-Fock method. Taken from my simulations for a course at the University of Oslo.}{
\includegraphics[width=0.48\linewidth]{../pictures/h2o_hf_density.png}}
\end{minipage}
\caption{Two pictures of the water molecule.}
\label{fig:water_molecule}
\end{figure}

\begin{table}[h!tb]
\caption{Experimental data for the water molecule.}
\label{tb:intro:h2odata}
\begin{center}
\begin{tabular}{c|c|c}
Description & Symbol & Value \\
\hline
H-O-H angle & $\theta$ & \SI{104.52}{\degree} \\
Distance O-H & $d_{OH}$ & \SI{0.9572}{\angstrom} \\
\end{tabular}
\end{center}
\end{table}

\section{Molecular dynamics modeling of methane hydrates}
Due to the lack of good experimental results, numerical modeling can be important to investigate methane hydrates. One of the ways to model them, is through molecular dynamics simulation, where trajectories of individual atoms are calculated using classical equations of motion. 

There are essentially two choices to be made when designing a molecular dynamics simulation: What potential models to use, and what system to simulate. By system, I mean both the initial condition and the conditions during simulation (temperature, pressure, etc.).

For the potential models of methane hydrates, there are two common strategies: All atom potentials and united atom potentials. In the all atom potentials, methane and water are represented by all of its atoms. Interactions between atoms belonging to the same molecule are bonding, and interactions between atoms belonging to different atoms are non-bonding. In the united atom potentials, all atoms in a molecule are represented by one particle. Interactions between molecules in united atom representations are non-bonding, and the functional form of the potential is usually more complicated than for the all atom models. 

In the first successful simulation of methane hydrate nucleation, \citet{Walsh2009} used a combination of these strategies. The water model was an all-atom model, TIP4P/Ice (described in detail in chapter \ref{ch:models}), and the methane model was a united-atom model, United-atom methane, which is a pure Lennard-Jones model. Figure \ref{fig:methane_hydrate_growth} shows illustrations of the formation process.  Later, \citet{Jacobson2010b} proposed a coarse-grained model using the united-atom approach both on the water model and the methane model. This model has later been used in several studies looking at nucleation and growth of methane hydrates. 

\begin{figure}
\centering
\includegraphics[width=\textwidth]{../pictures/methane_hydrate_growth.pdf}
\caption{Nucleation and growth of methane hydrates. From \citet{Walsh2009}. Reprinted with permission from AAAS.}
\label{fig:methane_hydrate_growth}
\end{figure}

\citet{Ning2010} calculate the bulk modulus of sI methane hydrates using several water models and methane models.

\citet{English2015} review molecular modeling of methane hydrates. Still no fracture. 

Chill+ identification of clathrate cages \citet{Nguyen}.

\section{Quantum mechanical calculations on methane hydrates}

\section{Molecular dynamics modeling of fracture}
\citet{doi:10.1142/9789812773326_0001} review some aspects of molecular modeling of fracture. 


\section{Research questions}
I will name a few outstanding quiestions concerning methane hydrates. @FindFromMechPropPaper.

\begin{enumerate}
\item What is the fracture toughness of methane hydrate? Fracture toughness of matarials are usually estimated with standardized mechanical tests. Since such tests are hard to do for methane hydrates, it is actually possible that the best estimates can come from molecular simulations. 
\item What characterizes failure in methane hydrates, is it ductile or brittle? Are there sub-critical fractures, and are they available on the timescale of molecular dynamics?
\item What does the fracture surface look like?
\item Are the fracture properties of methane hydrates predictable? It could for instance be that for a given stress or strain applied to a piece of methane hydrate, the critial stress/strain or the time it takes before it starts cracking is very predictable, and sharply peaked around some value. It can just as well be that statisical processes in the material are important, and that there is a wide waiting-time distribution governing the fracture initiation -- and even propagation.
\item Strain hardening to 20 \% strain. How can this be explained?
\end{enumerate}

There are also questions regarding how to model methane hydrates, and event within molecular dynamics modeling of methane hydrates, little is actually known -- especially when it comes to fracture. 

\begin{enumerate}
\item What potential best reproduce fracture of methane hydrates?
\item How may cracks be triggered in simulation if the results are going to say someting about reality?
\end{enumerate}

I will not end up answering many of these questions. I will, however, produce novel results and insights regarding molecular dynamics modeling of methane hydrate fractures.
@TellWhatQuestionsIEndUpAnswering