\begin{abstract}
\addcontentsline{toc}{chapter}{Abstract}
Elastic and failure properties of methane hydrates are studied using molecular dynamics simulations. The TIP4P/Ice water model and the OPLS united atom methane model are employed in the study. Mechanical properties are reported, and a possible fracture initiation process is identified. On the nanosecond timescale, a pure sI methane hydrate is identified as brittle, and with a fracture toughness of $\approx 0.06$ MPam$^\frac{1}{2}$. The initiation of cracks in the modeled systems is highly dependent on slow dissociation (melting) of the hydrate prior to rapid crack propagation. This meling occurs on the surface of initial and artificial flaws introduced in the hydrate. Furthermore, methane is immediately released upon fracture, while water molecules stick to the crack walls. This work provides some first steps into molecular dynamics studies of fracture in methane hydrates.
\end{abstract}