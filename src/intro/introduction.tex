% -*- root: ../main.tex -*-
\chapter{Introduction}
\section{The art of computer simulations}
The abundance of cheap computing resources developing over the last few decades has resulted in extensive computational studies in all parts of science. These studies have aided the understanding of physical phenomena, and the developent of new technologies. An early example is the atomic bomb. One of the immidiate applications of computer simulations, of which the results are used every day, is weather forecasting. Increasing access to high performace computers have resulted in weather forecasts that are useful several days in advance. 

A physical model, in the sense that I will be using it, is a mathematical formulation of a physical problem with the purpose of capturing some aspect of the behavior of that system. A simulation is a solution of the equations defining the model. For some simple models, the equations can be solved analytically, and in that case, the solution can be thought of as a simulation. For many models, we are not able to solve the equations analytically, and numerical methods must be applied to simulate the model -- this is what is usally referred to as a simulation.

Computer simulations are superior when it comes to having the full state of a system available in all relevant timescales and spatial scales, since these are available by design. That means simulations can aid understanding and predicting the behavoir of systems on scales that are not experimentally available. At the same time, computer simulations -- and physical models -- exist in close interplay with experiments. For a simulation of a model to be valueable, it has to reproduce relevant features of the physical system. These features are usually found in experiments. When a models is sufficiently calibrated from experimental data, it is possible -- with some uncertainty -- to use that model to study features that were not available from the experiment, and to propose new experiments that can shed light on the findings from the model, and possibly increase confidence in that model.

Computer models is a new thing, in the sense that running simulations on electronic computers was impossible before such devices were available. On the other hand, computer models are an old thing, since mathematical models could be formulated for computers long before electronic computers were even proposed. Indeed, iterative schemes like Newtons method for zero-points of a function and Eulers method for the integration of differential equations were developed by people who never lived to see electronic computers. The computations would be performed by a human, who would at the time be referred to as a computer. 

In the interception between computational science and theoretical physics, we find the field of computational physics: Mathematical models are developed to answer physics questions, and these models are studied using methods from computational science. On the other hand, computational physics is a close relative of experimental physics, in the way that the computational physicist perform ``numerical experiments'' when running simulations. But, the simulations are fully decoupled from the real world -- they are not real experiments. Thus the computational physicist is a theoretical physicist using a computer. Experimental intuition remains important.

\section{Methane hydrates}
Methane hydrates are so called clathrate compunds, meaning that they are consist of host molecules forming a lattice that traps guest molecules. For the case of methane hydrates, water molecules form cages that can host single methane molecules. At first sight, a piece of methane hydrate will resemble regular ice. However, if kept at regular temperatures and pressures, it will release methane gas, allowing it to be set on fire -- fiery ice. 

After multiphase piplelines were introduced in the petroleum industry, finding the physical conditions at which methane hydrates can form has been of great interest. Multiphase pipelines transporting oil, natural gas and water at the same time, are at risk of providing conditions at which methane hydrates can form and plug the pipeline. 

Over the last 10-20 years, rising prices on fossil fuels have started a revolution of research, exploration and recovery of unconventional hydrocarbon resources. Indeed, more than @percent of all natural gas recovered in the United States in @year came from unconventional sources, compared to @percent in @year @citation. 

Methane hydrates have not gotten nearly as much attention as the shale hydrocarbons, despite the estimated amounts of hydrocarbon residing in methane hydrates are estimated to be being significantly larger than those of shale hydrocarbons @citation. 
(Transition to gas hydrates being difficult to exptract because we don't know the dissociation mechanisms.)

Methane hydrate research has been boosted by the general hunt for unconventional sources of hydrocarbons. If methane production from methane hydrated turns out to be commercially viable, ...

(Selvforsterkende metan-apokalypse, synkehull i russland)

The goal for my master project is to study dissociation mechanisms for methane hydrates on the molecular scale. I will use molecular dynamics to study dissociation through crack propagation due to externally applied stress in pure, crystalline methane hydrates with artificial defects. The aim is to reproduce some simple mechanical and fracture properties in these systems, but not to reproduce or explain any particular experiments or phenomena. Additionally, some possibly fruitful paths for future research will be identified.

\section{The ethics of petroleum research}
Methane hydrates can be studied with the sole purpose of understanding their basic behavior, offering no attention to how that understanding might be used. But the reality is that if the political climate don't change during the coming years, knowledge about new hydrocarbons can result in higher CO$_2$-emmisions. Therefore, the ethics of petroleum research need to be adressed.

The conclusions of the Intergovernmental Panel on Climate Change (IPCC) states in its synthesis report from 2014 \cite{IPCC2014} that:

\begin{quotation}
The IPCC is now 95 percent certain that humans are the main cause of current global warming. In addition, the SYR finds that the more human activities disrupt the climate, the greater the risks of severe, pervasive and irreversible impacts for people and ecosystems, and long-lasting changes in all components of the climate system. 
\end{quotation}

This means that we cannot continue burning fossil fules at the rates that we have been doing over the last decade. A question naturally arising in the context of research, is whether it is responsible to do research that can potentially increase CO$_2$-emissions by making it easier to extract fossil fuels.

Fossil fuel dependence must by definition end at some point. Either by replacement for someting else, or by fossil fuels turning out not to be fossil (which is highly unlikely). But fossil fuel extraction cannot stop tomorrow. If that happened, people would start dying, since energy-intensive fossil-fueled activity is essential for food production, healthcare, transportation and warming houses in cold areas.

Fossil fuels can be ranked by how much CO$_2$ they emit per Joule of energy produced.

Technologies for carbon capture and storage (CCS) are under development, and if they succeed, fossil fuel extraction can continue beyond the amount that is safe to burn and release into the atmosphere. 

Gas is better than coal -- but it is not good enough.

Whether fossil fuel extration and burning continue within safe amounts or not, it \emph{will} continue for several decades, and research on the topic will contribute to knowledge on fossil fuels. If that knowledge is used responsibly, that knowledge is good.



I have concluded that I can do research on petroleum. 

\section{My contribution}
A master thesis by 2014 (Cite kriteriene? :P), by design, requires the student to describe rather that reference known theory. This feature of the masters thesis calls for a detailed account of what are my scientific contributions. I have investigated the TIP4P/ICE molecular water model, and worked out previously unknown properties of that model, includeing the viscosity and diffusion coefficient. I have studied fracture of methane hydrates in a molecular dynamics model, which to the best of my knowledge, no one did before. 

I have developed some analysis tools, which are described in chapter \ref{ch:tools}. It should be clear what parts of that chapter are my descriptions of tools made by others, and what tools were developed by me.

Most of my work has been on the properties of methane hydrates in the TIP4P/ICE water and OPLS-UAM methane (explained in detail in chapter \ref{ch:models}). This model, like any molecular model of methane hydrates, is poorly investigated. I have made progress on the following:

\begin{itemize}
\item Mechanical properties: Youngs modulus, Poissons ratio.
\item Fracture toughness under tensile strain.
\item Fracture propagation under tensile strain.
\end{itemize}


\section{What is the master thesis?}
This thesis is a walkthrough of how to do molecular simulations, how to apply it to methane hydrates, and how to verify and interpret the results. In part, it is also a documentation of my work during the last year. If I have been strugling with a topic, and spent a lot of time figuring it out, this should be reflected in the thesis. Therefore, these parts will be more elaborate than other parts of the thesis. The section proof-of-concept runs in the chapter about modeled systems repserents a lot of work to find the right parameters to go into the model, and it is also the part of the project where I met a bug in LAMMPS that took a lot of time to try to get around, before I eventually decided not to use the buggy part of LAMMPS.


\section{Structure of this thesis}

Chapter \ref{ch:state_of_the_science} sums up relevant aspects of the state of the science of methane hydrates, both experimentally and theoretically, and ends with questions that I try to answer in my research.
...
In part I is also a presentation of the theory of molecular dynamics. I have sought to explain all theory needed to in principle implement a molecular dynamics program that would do the same job as the one I use. That means I aim to present an overview of what is important for the physics of the system, but not implementation details that are unimportant to the actual outcome of my simulations. Some topics can be explained in detail, while some are too technical, requiring too many details to be interesting in a master thesis. I will try to refer to research papers that can supplement the presentation where my presentation is incomplete.
...
Part II is the main part of this thesis. Chapter \ref{ch:models} contains the choice of mathematical model for the methane hydrates in addition to a description of the system that I will study. Chapters \ref{ch:tools} and \ref{ch:verification} describes the numerical tools i use and the tools I have developed, along with verification of the numerics. Chapter \ref{ch:modeledsystems} contains the main results. This chapter is mostly organized chronologically after what time I did what, and it can be seen as a description of the research process that I have been going through the last months. Some tools were devoloped during the research process, and these tools are described in detail in chapter \ref{ch:tools}, but it should be clear from the discussion in \ref{ch:modeledsystems} when they were actually introduced.


