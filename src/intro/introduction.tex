% -*- root: ../main.tex -*-
\chapter{Introduction}
In this chapter, I start by introducing computer simulations and methane hydrates, and discuss the ethics of hydrocarbon research. Then i go on to list the contributions of my work, the goals and outcomes of the project, and describe the structure of this thesis. The methane hydrate introduction in this chapter is brief, and a more thorough description is given in chapter \ref{ch:state_of_the_science}.

\section{The art of computer simulations}
The abundance of cheap computing resources developing over the last few decades has resulted in extensive computational studies in all parts of science. These studies have aided the understanding of physical phenomena, and the development of new technologies. An early example is the atomic bomb. An everyday example of the application of computer simulations, is weather forecasting. Increasing access to high performance computers have resulted in weather forecasts that are useful several days in advance. In the first example -- the atomic bomb -- computer simulations aided physical understanding. In weather forecasting however, we already know the physics. The model incorporates known laws of nature to predict weather conditions forward in time. I will be doing the former -- but on another topic.

A physical model, in the sense that I will be using it, is a mathematical formulation of a physical problem with the purpose of capturing some aspect of the behavior of that system. A simulation is a solution of the equations defining the model. For some simple models, the equations can be solved analytically, and in that case, the solution can be thought of as a simulation. For many models, we are not able to solve the equations analytically, and numerical methods must be applied to simulate the model -- this is what is usually referred to as a simulation.

Computer simulations are superior when it comes to having the full state of a system available in all the timescales and spatial scales of the simulation, since these are available by design. That means simulations can aid understanding and predicting the behavior of systems on scales that are not experimentally available. Conversely, a computer simulation is \emph{always} limited in time and space, since computing power is finite. First-principles studies can only be done on small and short scales. To do simulations on larger and longer scales, details must be omitted, resulting in approximated models. The quality of these approximations are crucial, and the profound implication is that it is hard to know whether a simulation captures the relevant physics. 

The approximations of the models represent a substantial difference between simulations and physical experiments. At the same time, computer simulations and physical models exist in close interplay with experiments. For a simulation of a model to be valuable, it has to reproduce relevant features of the physical system. These features are usually found in experiments. When a models is sufficiently calibrated from experimental data, it is possible -- with some uncertainty -- to use that model to study features that were not available from the experiment, and to propose new experiments that can shed light on the findings from the model, and possibly increase confidence in that model.

Computer models are a new thing, in the sense that running simulations on electronic computers was impossible before such devices were available. On the other hand, computer models are an old thing, since mathematical models could be formulated for computers long before electronic computers were even proposed. Indeed, iterative schemes like Newton's method for roots of a function and Euler's method for the integration of differential equations were developed by people who never lived to see electronic computers. The computations would be performed by a human, who would at the time be referred to as a computer. 

In the intersection between computational science and theoretical physics, we find the field of computational physics: Mathematical models are developed to answer physics questions, and these models are studied using methods from computational science. On the other hand, computational physics is a close relative of experimental physics, in the way that the computational physicist perform ``numerical experiments'' when running simulations. However, the simulations are fully decoupled from the sensory world -- they are not real experiments. Thus the computational physicist is a theoretical physicist using a computer. Still, experimental intuition remains important, as computer simulations resemble experiments, and the analysis methods are similar.

%%%%%%%%%%%%%%%%%%%%%%%%%%%%%%%%%%%%
\section{Methane hydrates}
%%%%%%%%%%%%%%%%%%%%%%%%%%%%%%%%%%%%
Methane hydrates are so called clathrate compounds, meaning that they consist of host molecules forming a lattice that traps guest molecules. Methane hydrates are a special case of \emph{gas hydrates}, or \emph{clathrate hydrates} which they are also called. In methane hydrates, water molecules form cages that can host single methane molecules. At first sight, a piece of methane hydrate will resemble regular ice. However, if kept at regular temperatures and pressures, it will release methane gas, allowing it to be set on fire -- fiery ice. 

When multiphase pipelines were introduced in the petroleum industry, finding the necessary physical conditions for methane hydrate precipitation became interesting and necessary. Multiphase pipelines -- pipelines that transport oil, natural gas and water at the same time -- are at risk of providing conditions at which methane hydrates can form a plug that will decrease or stop flow in the pipeline, and possibly damage equipment. The first observation of such plugs were done in the 1930's \cite{Hammerschmidt1934}.

Over the last 10-20 years, rising prices on fossil fuels have started a revolution of research, exploration and recovery of unconventional hydrocarbon resources. Indeed, around 50 \% of all natural gas and oil recovered in the United States in 2014 came from hydrocarbon bearing shales \cite{EIA2015}.

Methane hydrates have not received nearly as much attention as shale gas, despite the estimated amounts of hydrocarbon residing in methane hydrates being significantly larger than those of shale hydrocarbons. That might be because methane hydrates seem harder to extract. Gas hydrate extraction is still in the piloting phase, and it is still not commercially viable.

In addition to the interest in methane hydrates as an energy resource, is also the fear that methane hydrates can -- if released -- contribute significantly to out-of-control global warming. Methane hydrates are stable within some range of temperatures and pressures, but if brought out of that range, the dissociate, and release the methane. Methane is a powerful greenhouse gas, and the idea is that the released methane will contribute to even more warming, releasing more methane in a process spinning out of control: \emph{The clathrate gun hypothesis} \cite{kennett2003methane}. This hypothesis is set both as a possible explanation of past events of global warming (on geological timescales), and a fear for the future.

Understanding the dissociation mechanisms of methane hydrates is important both with respect to extraction and with respect to the possibly hazardous behavior of methane hydrates during global warming. 

\section{The ethics of hydrocarbon research}
Methane hydrates can be studied with the sole purpose of understanding their basic behavior, offering no attention to how that understanding might be used. But the reality is that if the political climate doesn't change during the coming years, knowledge about new hydrocarbons can result in higher CO$_2$-emissions. Therefore, the ethics of petroleum research need to be addressed.

The Intergovernmental Panel on Climate Change (IPCC) concludes in its synthesis report from 2014 \cite{IPCC2014} that:

\begin{quotation}
The IPCC is now 95 percent certain that humans are the main cause of current global warming. In addition, the SYR finds that the more human activities disrupt the climate, the greater the risks of severe, pervasive and irreversible impacts for people and ecosystems, and long-lasting changes in all components of the climate system. 
\end{quotation}
This means that we cannot continue burning fossil fuels at the rates that we have been doing over the last decade. A question naturally arising from that, is whether it is responsible to do research that can potentially increase CO$_2$-emissions by making it easier to extract fossil fuels.

Fossil fuel dependence must by definition end at some point, but fossil fuel extraction cannot stop tomorrow. That would have a great impact on peoples lives and standard of living, and an abrupt discontinuation of fossil fuel extraction would ultimately lead to higher mortality rates.

Fossil fuels can be ranked by how much CO$_2$ they emit per Joule of energy produced. In that respect, it would for instance be better to research gas than coal. But gas is not good enough, unless it is only a bridge towards a fossil-free world.

Technologies for carbon capture and storage (CCS) are under development, and if they succeed, climate is no longer the limiting factor for using fossil fuels. Then the game is suddenly changed, and research that enables extraction of previously inviable resources.

The Norwegian National Committee for Research Ethics in Science and Technology (NENT) conducted in 2014 an assessment of the ethics of Norwegian petroleum research. Their main conclusion, as stated on their web page is \cite{NENT2014}:

\begin{quotation}
It is indefensible from a research ethics perspective if petroleum research hinders processes of transition to sustainable energy and thus prevents achievement of UN climate goals which Norway has pledged to uphold
\end{quotation}
This is not really a conclusion, since the premise is still very much up for debate. Does petroleum research hinder the transition to sustainable energy? These questions are still open and under debate. 
Here, I conclude that there \emph{are} ethical concerns regarding petroleum research, but it is not obvious or how to answer or resolve them in a best possible way for humankind.

\section{My contribution}
This thesis contains descriptions of previous works and original contributions.

I have developed new tools for analysis, which are described in chapter \ref{ch:tools}. It should be clear from the description what parts of that chapter are my descriptions of tools made by others, and what tools were developed by me.

Most of my work has been on the properties of methane hydrates in the TIP4P/ICE water + OPLS-UAM methane model (explained in detail in chapter \ref{ch:models}). This model, like all molecular dynamics models of methane hydrates, is poorly investigated. Additionally, fracture of methane hydrates is poorly studied, and to my knowledge, this is the first study of fracture of methane hydrates using molecular dynamics. 

Below follows a list of my scientific contributions in this thesis:

\begin{itemize}
\item Transport properties of the TIP4P/Ice water model at \SI{300}{\kelvin}. (Diffusion coefficient and viscosity).
\item Mechanical properties of methane hydrates in TIP4P/Ice+OPLS-UAM: Young's modulus, Poisson's ratio.
\item Fracture toughness of TIP4P/Ice + OPLS-UAM under tensile strain.
\item Brittleness of TIP4P/Ice+OPLS-UAM under rapid loading.
\item Separation of two distinct stages of fracture propagation in TIP4P/ICE+OPLS+UAM. 
\end{itemize}

\section{Goals and outcomes}
The goal for my master project is to study dissociation mechanisms for methane hydrates on the molecular scale. I will use molecular dynamics to study dissociation through crack propagation due to externally applied stress in pure, crystalline methane hydrates with artificial defects. The aim is to reproduce some simple mechanical and fracture properties in these systems, but not to reproduce or explain any particular experiments or phenomena. Additionally, some possibly fruitful paths for future research will be identified.

The project was initially outlined with the following sub-projects:
\begin{enumerate}
\item \tb{Develop molecular dynamics models for gas hydrate modeling:}
The student will apply standard molecular dynamics codes, such as LAMMPS,
to study gas hydrates in the stable regime. In addition, we will develop our
own codes with reactive potentials, and validate the model by comparison with
standard codes in the stable regime.
\item \tb{Apply model to study gas hydrate dissociation:}
The student will use the validated models to address gas hydrate dissociation
processes in scenarios relevant for production.
\item \tb{Refine model to include sedimentary interactions:}
Finally, the student will develop the model to include effects of sediments, such
as by including effects of silicate glasses for which we have well-tested reactive
potentials, to address the effect of the surrounding sedimentary minerals and
confined pore spaces on gas hydrate dissociation processes.
\end{enumerate}

During the work on the first subproject, we decided that time could be better spent studying fracture as a dissociation mechanism in methane hydrates, since this kind of study has been proposed but not yet conducted. So in reality, the project has consisted of the following sub-projects:

\begin{enumerate}
\item \tb{Develop molecular dynamics models for gas hydrate modeling:}
The student will apply standard molecular dynamics codes, such as LAMMPS,
to study gas hydrates in the stable regime.
\item \tb{Apply model to find a protocol to study fracture of methane hydrates:}
Use the validated model to find a way to impose and analyze fracture in the methane hydrate model. 
\item \tb{Refine the protocol for studying fracture, and characterize fracture in methane hydrates:}
Use the developed protocol to run simulations and study dissociation and mechanisms for fracture initiation and propagation in methane hydrates.
\end{enumerate}

Note that the project no longer contains the development of our own code. During the work on the first subproject, LAMMPS was found sufficiently flexible for developing our own potentials within its framework. This has not been done in this work because the fracture study was prioritized, not leaving time for the development of our own reactive potentials. But reactive potentials will be a priority in future work.

\section{Structure of this thesis}
In addition to being a scientific record, this thesis is also a documentation of my work during the last year. If I have been struggling with a topic, and spent a lot of time figuring it out, this will be reflected in the thesis. These parts will be more elaborate than other parts of the thesis, regardless the scientific value of their contents. An example is the section ``proof-of-concept runs'' in the chapter about modeled systems.  This section represents a lot of work to find the right parameters to go into the model, but the results have little impact beyond the validation of the model and it's applicability to the problems I work with.

This thesis is arranged in four parts. Part I contains background chapters. It starts with chapter \ref{ch:state_of_the_science} on methane hydrates, which sums up relevant aspects of the state of the science of methane hydrates, both experimentally and theoretically, and ends with questions that my research aims to answer. Then comes two theory chapters: one on elasticity and failure, and one on molecular dynamics. I have sought to explain theory needed to for understanding the physics of a molecular dynamics simulation. That means implementation details that are unimportant to the actual outcome of my simulations are not discussed. Some topics are explained in detail, while others are too technical, requiring too many details to be interesting in a master thesis. In these cases I will refer to research papers that can supplement my presentation. Part II is the main part of this thesis. In chapter \ref{ch:models} I introduce interaction potentials, and choose which ones to use to describe methane hydrates. In addition, I describe the system that I will study (Initial particle positions, thermodynamic conditions and boundary conditions). Chapters \ref{ch:tools} and \ref{ch:verification} describes the numerical tools I utilize and the tools I have developed, along with a brief verification of the numerics. Chapter \ref{ch:modeledsystems} contains the main work and results. This chapter is mostly organized chronologically after when I did what, and it can be seen as a description of the research process that I have been going through after I got the simulation tools to work properly. Some tools were developed during the research process, and these tools are described in detail in chapter \ref{ch:tools}, but it should be clear from the discussion in \ref{ch:modeledsystems} when they were actually introduced. Part III contains only one chapter, chapter \ref{ch:summary_conclusions}, where I summarize my results, conclude based on those results, and propose possible topics for future work. Part IV contains the appendices: Contents that have value, but would hamper the flow of the document if they were presented in the main text.


