% -*- root: ../main.tex -*-
\chapter{Introduction}
\section{Models, experiments and reality}
Some take the position that 

( I may want to cite George E. P. Box - ``All models are wrong, but some are useful'')

\section{Methane hydrates}
Methane hydrates are so called clathrate compunds, meaning that they are consist of host molecules forming a lattice that traps guest molecules. For the case of methane hydrates, water molecules form cages that can host single methane molecules. At first sight, a slab of methane hydrate will resemble regular ice. However, if kept at regular temperatures and pressures, the slab will release methane gas, allowing it to be set on fire; fiery ice. 

(Fossil fuel industry; clogged pipelines and vast energy reserves)
In the past, methane hydrate formation has been of great interest, because multiphase pipelines transporting oil, natural gas (methane) and water tend to get clogged by water and methane forming methane hydrates. Over the last 10-20 years however, rising prices on fossil fuels have started a revolution of research, exploration and recovery of unconventional hydrocarbon resources. Indeed, more than @percent of all natural gas recovered in the United States in @year came from unconventional sources, compared to @percent in @year @citation. 

(Transition to gas hydrates being difficult to exptract because we don't know the dissociation mechanisms.)

Methane hydrate research has been boosted by the general hunt for unconventional sources of hydrocarbons. If methane production from methane hydrated turns out to be commercially viable, ...

(Selvforsterkende metan-apokalypse)

The mission for my master project is to investigate dissociation mechanisms for methane hydrates on the molecular scale. The choices made by me and my supervisor have focused the project into using molecular dynamics to study dissociation through critical and sub-critical crack formation due to externally applied stress in pure and regular methane hydrates with artificial defects. 



\section{My contribution}
A master thesis by 2014 (Cite kriteriene? :P), by design, requires the student to describe rather that reference known theory. This feature of the masters thesis calls for a detailed account of what are my scientific contributions. I have investigated the TIP4P/ICE molecular water model, and worked out previously unknown properties of that model, includeing the viscosity and diffusion coefficient.

\section{Structure of this thesis}
...
In part I is also a presentation of the theory of molecular dynamics. I have sought to explain all theory needed to in principle implement a molecular dynamics program that would do the same job as the one I use. That means I aim to capture all that is important for the physics of the system, but not implementation details that are unimportant to the actual outcome of my simulations. (to machine precision). (Example: I explain PPPM but not how to slit the domain for parallellization).
...
